\section{Applications of ACO to other problem types}
ACO algorithm can be applied to other problem types either directly or by approximated way.

\begin{itemize}
\item \textbf{Continuous Optimization Problems}. Simplest way is to approximate the problem to its discrete analogue, however it cannot be applied to certain problems. Some adaptations are carried out by Socha and Dorigo \cite{aco_continuous}. Another model was presented by Liao\cite{aco_incremental}.

\item \textbf{Multi-objective problems} are the problems that have several defined objective functions either have a deterministic preference model or demand the Pareto set as the solution. In practice such problems normally do not have a preference model over the pareto-optimal solutions. Several algorithms for resolving such problems were reviewed in the paper of L{\'o}pez-Ib{\'a}{\~n}ez and Thomas St{\"u}tzle \cite{moaco}. In that work a MOACO framework has been described which operated 10 multi-objective parameters and 12 ACO parameters. There are two common ways: maximizing of each of objectives (COMPETants) or finding non-dominated solutions (BicreterionAnt, MACS, mACO-3). The feature of MOACO framework is that it stores multiple pheromone trail data for each of the specified objectives. In the same time the weighted sum of the objective functions' values composes a unified scalar objective function.

\item \textbf{Dynamic problems} are the problems that have some information revealed during the execution. These problems are resolved in a strongly different manner - ants act asynchronously and no global pheromones are updated, instead specific update mechanisms are held.

\item \textbf{Stochastic problems} deal with information that is not deterministic. Such problems as probabilistic TSP where for each city there is a given probability that it is required to visit.
\end{itemize}