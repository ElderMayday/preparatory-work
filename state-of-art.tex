\documentclass[12pt]{article}
\usepackage{caption}
\usepackage{float}
\usepackage{hyperref}
\usepackage{xcolor}
\usepackage{mdframed}
\usepackage{tabto}
\usepackage{amsmath}
\author{Aldar Saranov}
\date{\today}
\title{Development of an automatically configurable ant colony optimization framework. State of art.}

\newmdenv[
  backgroundcolor=gray!20,
  frametitle=Definition,
  skipabove=\topsep,
  skipbelow=\topsep,
]{definition}

\newmdenv[
  backgroundcolor=white!20,
  frametitle=Algorithm,
  skipabove=\topsep,
  skipbelow=\topsep,
]{algorithm}

\newcommand{\dd}[1]{\mathrm{d}#1}


%-----------------------------------------------------

\begin{document}

\maketitle 
\newpage

\tableofcontents
\newpage

\begin{abstract}
Some animal species show an extreme degree of social organization. Such species (e.g. ants) have pheromone production and detection body parts and therefore seize an ability to communicate between each other in indirect way. This concept has inspired the development of algorithms which are based on social behavior of population called ant colony optimization algorithms (ACO). These algorithms allow to solve NP-hard problems in a very efficient manner. Since these algorithms are considered metaheuristic the development of a ACO framework is the next step of formalizing of this area is to provide tools for resolving general optimization problems. This article gives the brief overview of the current ACO research area state, existing framework description and some tools which can be used for the framework automatic configuration.
\end{abstract}



\section{Introduction (1 page)}
Section descriptions.
Pheromones.
Constructive heuristics.
Solution components.
Problem models.

\section{Combinatorial Optimization Problems and Constructive
Heuristics (11 pages)}

Combinatorial optimization problems (COPs) are a class of mathematical optimization problems. These problems can be described as a grouping, ordering, assignment or any other operations over a set of discrete objects. In practice, one may need to resolve COPs, which have a large number of extra constraints for the solutions to consider them as admissible. Many of these problems which have been thoroughly researched at the moment, belong to NP-hard discrete optimization problems. NP-hard problem means that we cannot decompose a large instance of such problems into a smaller one. The best performing algorithms known today solve such problems have a worst-case run-time larger than polynomial (e.g. exponential).

\begin{minipage}[c, breaklines=true]{0.95\textwidth}
\begin{definition}
	\underline{An Optimization Problem} is a tuple ($\Phi,\omega, f$), where
	\begin{itemize}
		\item{$\Phi$ is a \underline{search space} consisting of all possible assignments of discrete variables $x_i$, with $i=1,...,n$ }
		\item{$\omega$ is a \underline{set of constraints} on the decision variables}
		\item{$f:\Phi \to R$ is an \underline{objective function}, which has to be optimized}
	\end{itemize}
\end{definition}
\end{minipage} \\

The problem describes the abstract task (e.g. find the minimum spanning tree of some graph), while the instance of a problem describes a specific practical problem (e.g. find the minimum spanning tree of a given graph G). The objective function is the sum of the weights of the selected edges. \\
One of the most studied problems is the traveling salesman problem (TSP). Given a graph $G=(N,E)$ with $n=|N|$ nodes and $E$ being the set of edges that fully connects the nodes and distances $d_{ij}, \forall(i,j) \in E$, one should find a Hamiltonian path of minimal length (in terms of sum of the weighted edges). The solution path can be represented as permutation $\pi=(\pi_1,...,\pi_n)^t$ of all \emph{n} nodes, where $\pi_i$ is the node index at position i. The objective is

\begin{equation}
\min \limits_{\pi \in \Phi} {d_{\pi_i \pi_{i+1}}} + \sum \limits_{i+1}^{n-1} {d_{\pi_i \pi_{i+1}}}
\end{equation}

Thus $\pi$ forms a permutation space and every permutation of $\pi$ gives a admissible (but not necessarily optimal) solution. It is obvious that the absolute position in the permutation sequence does not affect the value of the objective function but the relative position.

In addition to the TSP, the quadratic assignment problem (QAP) was deeply researched. In the QAP, there is a set of $n$ locations and a set of $n$ facilities which are connected by flows. An instance of the problem is given by $n \times n$ matrices $(d_{ij})$ and $(f_{ij})$ with $i, j = 1,...,n$. A solution of the QAP is an assignment of the facilities to the locations represented by permutation $\pi$ where $\pi_i$ depicts the facility that is assigned to the location $i$. The objective function is represented as the sum of paired products of distances between $i$ and $j$ locations and specified flows between $\pi_i$ and $\pi_j$ assigned flows.

\begin{equation}
\min \limits_{\pi \in \Phi} \sum \limits_{i=1}^n \sum \limits_{j=1}^n {d_{ij} \times f_{\pi_i \pi_j}}
\end{equation}

Solution components are normally defined in the terms of the COPs to be solved. Solution components $C=\left\{c_1,c_2,...\right\}$ is a set, subset of which corresponds to one solution of the given problem (if it also fulfills the constraints). Solutions that fulfill all constraints are also called \emph{feasible solutions}. In the case of the TSP, solution components are the edges $(i,j)$ of the given graph. In the case of the QAP, solution components are all the possible assignments of every location $i$ to every facility $j$. In order to provide feasible solutions, the algorithm must either operate completely in the feasible candidate solution space or bias towards the feasible ones with final constraint checking. \\

Since solving of such NP-hard problems by trying to find provably optimal solutions is unreasonable one can apply heuristic algorithms, which more or less provide solutions with relatively good quality consuming reasonable quantity of resources (time/power, memory etc.). \\

An important class of such heuristic algorithms are constructive heuristics. Constructive heuristics start with an empty or partially built solution, which is then being completed by iterative extension until a full solution is completed. Each of the iterations adds one or several solution components to the solution. For example, greedy constructive heuristics  add the best-ranked components by their heuristic values. \\

For the TSP, the \emph{nearest neighbor} heuristic can be used. The nearest neighbor heuristic starts from a random node $\pi_i$ with initial random $\pi=<\pi_1>$. At each step it selects the node with the minimal distance $d_{ij}$ to the current node and adds the corresponding next node $\pi_2=j$ components to the solution. The index of the next node at each step can be computed as following.\\

\begin{equation}
nextNode = \operatornamewithlimits{argmin} \limits_{j \in N, j \notin \pi} {(d_{ij})}
\end{equation}

For the QAP, one tends to place the facilities which exchange a lot of flow at the locations that are more "central". The algorithm computes $f=(f_1,...,f_n)^t$ where $f_i=\sum \limits_{j=1}^n {f_{ij}}$ and $d=(d_1,...,d_n)^t$ where $d_k=\sum \limits_{l=1}^n {d_{kl}}$. Then the algorithm assigns the facility with the largest $f_i$ to the location with smallest $d_k$ and iterates through these steps. \\

Generally heuristic values are assigned as constants at the start of the solution construction. However, in extensions one can use heuristic values, which are a function of the generated partial solution. This is called \emph{adaptive} heuristics and normally it consumes larger computer resources although leads to better quality of the solutions.
\section{The Concepts of Ant Colony Optimization}

\begin{algorithm}
\begin{tabbing}
\hspace*{1cm} \= \hspace*{1cm} \= \hspace*{1cm} \= \hspace*{1cm} \= \hspace*{1cm} \= \\
\textbf{procedure} ACO-Metaheuristic \\
\> \textbf{repeat} \\
\> \> \textbf{for each} ant \textbf{do} \\
\> \> \> \textbf{repeat} \\
\> \> \> \> ExtendPartialSolutionProbabilistically() \\
\> \> \> \textbf{until} solution is complete \\
\> \> \textbf{for each} ant $\in$ SelectAntsForLocalSearch() \textbf{do} \\
\> \> \> ApplyLocalSearch(ant) \\
\> \> EvaporatePheromones() \\
\> \> DepositPheromones() \\
\> \textbf{until} termination criteria met \\
\textbf{end}
\end{tabbing}
\end{algorithm}


\subsection{Choice of pheromone trails and heuristic information}

Generally there are two mechanisms of biasing the solution production - pheromones and heuristic values. \\
Hereby we introduce the following components: \\
$C$ - Solution components. \\
$\tau_c \in T$ - pheromones of choosing. \\
$\tau'_c \in T'$ - pheromones of considering order. \\
$\pi$ - candidate solution. \\
$\eta_c \in H$ - heuristic information (constant in time). \\

Higher values of $\tau_c$ stand for higher probability of that the component $c$ will be added to the solution. Additional problem-specific pheromones as $tau'_c$ are used for auxiliary purposes (e.g. desirability of considering of one facility after another in QAP). Heuristic information $H$ is similar to the pheromone trails however it is not updated during the algorithm execution ($\forall c \in C, \exists \eta_c \in H$). Those are either constant values or values which depend on the current partially constructed solution.

\subsection{Solution component choice}

Solution construction phase as says the name yields a new solution set. Each ant starts with an empty solution $s$. Each ant produces may produce one solution at one run. At each step one solution component is added. The probability of $c_j$ to be added at certain step can be calculated by different techniques (i.e. $Pr(c_j|T,H,s)$). Frequently used rule is defined as follows:

\begin{equation}
Pr(c_j)=\frac{t_j^\alpha \times \eta_j^\beta}{\sum \limits_{c_k \in N_i} t_k^\alpha \times \eta_k^\beta} \forall c_j \in N_i
\label{eq:construction_classic}
\end{equation}

$\alpha$ and $\beta$ are the parameters which determine the impact of the pheromone trails and heuristic information on the final probability. Another alternative has been proposed by Maniezzo [82,83] which combines the pheromone trails and heuristic information in a linear way.

\begin{equation}
Pr(c_j)=\frac{\alpha \times \tau_j + (1-\alpha) \times \eta_j}{\sum \limits_{c_k \in N_i} \alpha \times \tau_k + (1-\alpha) \times \tau_k} \forall c_j \in N_i
\end{equation}

Since it does not use exponentiation operations this algorithm is preferable for performance-targeted frameworks. However this algorithm may cause undesired biases if the range of the values are not taken into account. The third alternative is invented by Dorigo and Gambardella [34] with Ant Colony System (ACS) algorithm. This algorithm is also called pseudo-random proportional rule. A random uniform value $q$ is generated at range $[0;1)$ and if $q>q_0$ where $q_0$ is a predefined parameter then probability is being calculated according to the formula \eqref{eq:construction_classic}. Otherwise the solution component is picked as:

\begin{equation}
c_j = \operatornamewithlimits{argmax}\limits_{c_k \in N_i} t_k^\alpha \times \eta_k^\beta
\label{eq:construction_dorigo}
\end{equation}

Apparently larger $q_0$ gives more greedy choice.

\subsection{Construction extensions}

\textbf{Lookahead} conception was introduced. Is says that at each decision step several solution components should be considered at once in order to get the next solution component. Generally it is worth to be implemented when the cost of making a local prediction based on the current partial solution state is much lower than the cost of the real execution of the move sequence.

\textbf{Candidate list} restricts the solution component set to a smaller set to be considered. The solution components in this list have to be the most promising at the current step. Usually this approach yields a significant gain depending on the initial set-up (i.e. if this list is precalculated once before the run). Nonetheless it can also depend on the current partial solution. For TSP it is represented as nearest neighbor list for each of the cities.

\textbf{Hash table} of pheromone trails. It allows to efficiently save memory when the updated pheromone trails are is a sparse set in comparison to the set of all solution components. Search and updating of the elements of the hash-table is expected to be done within linear time.

\textbf{Heuristic precomputation} of the values $t_j^\alpha \times \eta_j^\beta$ for each of the solution components which are used in \eqref{eq:construction_dorigo}.

Extensions:
\begin{itemize}
\item {Lookahead - pick several components at once[94]}
\item {Candidate list - restriction of component choice at each step[33,34]}
\item {Iterated greedy (partial deconstruction)[110]}
\item {With external memory[1]}
\item {Iterated ants[129]}
\item {Cunning ants[128]}
\item {Enhanced ACO[47]}
\end{itemize}

\subsection{Global pheromone update}
Evaporation: \\
$\tau_{new}=evaporation(\tau_{old}, \rho, S^{eva})$ \\
$\rho$ - evaporation rate \\
$S^{eva}$ - chosen solutions for evaporation \\ \\

Deposition: \\
$w_k$ - weight of solution $s_k$. \\
$F(S_k)$ - non-decreasing solution quality scaling function. \\ \\

Update selection:
\begin{enumerate}
\item {Ant system (update all)}
\item {Single update selections:}
\begin{enumerate}
\item {iteration-based update}
\item {global-based update}
\item {restart-based update}
\end{enumerate}
\end{enumerate}

Update extensions: \\
\begin{enumerate}
\item {Max-Min Ant System [122]}
\item {Rank-based Ant System [19]}
\item {Best-Worst Ant System [21]}
\item {Elitist Ant System [30, 36, 38]}
\end{enumerate}

\subsection{Pheromone update schedule}
Exploration vs exploitation.

\subsection{Initialization of pheromones}


\subsection{Pheromone reinitialization}


\subsection{Local pheromone update}
Parallel vs sequential.
ACS [34]

\subsection{Pheromone limits}
MMAS and ACS examples.

\subsection{Local search}
Neighborhood operator.
Best-improving and first-improving.

\subsection{ACO algorithms as instantiations of the ACO Metaheuristc}

\section{Applications of ACO to other problem types}

\subsection{Continuous Optimization Problems}

\subsection{Multi-objective problems}

\subsection{Dynamic problems}

\subsection{Stochastic problems}

\section{ACO in combination with other methods}

\subsection{ACO and tree search methods}

\subsection{ACO and exact methods}

\subsection{ACO and surrogate models}

\subsection{Parameter adaptation}

\section{Existing ACO framework (5 pages)}

\subsection{Finding a better ACO configuration for the TSP}

\subsection{Finding a better ACO configuration for the QAP}

\section{IRACE automatic configuration (3 pages)}

\section{Conclusions}

\begin{thebibliography}{1}

\end{thebibliography}

\end{document} 
